\renewcommand{\familydefault}{\sfdefault}
\usepackage[textwidth=16cm,textheight=23cm]{geometry}
\setlength{\parindent}{6pt}
\setlength{\parskip}{8pt}
\usepackage{titlesec}

\usepackage{graphicx} % Required for inserting images
\usepackage{natbib}
\usepackage{amsmath}
\usepackage{amsfonts}
\usepackage{mathtools}
\usepackage{subcaption}
\usepackage{xcolor}
\usepackage{url}
\usepackage{ifthen}
\usepackage{wrapfig}
\usepackage{tcolorbox}
\tcbuselibrary{skins}

% Define citation color
\definecolor{citblue}{HTML}{2d576b}

% Define TUM corporate design colors
% Taken from http://portal.mytum.de/corporatedesign/index_print/vorlagen/index_farben
\definecolor{TUMBlue}{HTML}{0065BD}
\definecolor{TUMSecondaryBlue}{HTML}{005293}
\definecolor{TUMSecondaryBlue2}{HTML}{003359}
\definecolor{TUMBlack}{HTML}{000000}
\definecolor{TUMWhite}{HTML}{FFFFFF}
\definecolor{TUMDarkGray}{HTML}{333333}
\definecolor{TUMGray}{HTML}{808080}
\definecolor{TUMLightGray}{HTML}{CCCCC6}
\definecolor{TUMAccentGray}{HTML}{DAD7CB}
\definecolor{TUMAccentOrange}{HTML}{E37222}
\definecolor{TUMAccentGreen}{HTML}{A2AD00}
\definecolor{TUMAccentLightBlue}{HTML}{98C6EA}
\definecolor{TUMAccentBlue}{HTML}{64A0C8}

\newtcolorbox{mybox}{
    enhanced,
    frame hidden,
    borderline west = {0.5pt}{0pt}{black},
    boxrule=0pt,
    arc=0pt,
    outer arc=0pt,
    left=5pt,
    right=5pt,
    top=5pt,
    bottom=5pt,
    boxsep=0pt,
}

\NewDocumentCommand{\floatingbox}{ O{L} O{\textwidth} m }{%
    \ifthenelse{\equal{#2}{\textwidth}}{
        \begin{mybox}
            #3
        \end{mybox}
    }{
        \begin{wrapfigure}{#1}{#2}
            \begin{mybox}
                #3
            \end{mybox}
        \end{wrapfigure}
    }
}

% Redefine \citet command
\let\oldcitet\citet
\renewcommand{\citet}[1]{%
    \citeauthor{#1}%
    \ \textcolor{citblue}{[\citeyear{#1}]}%
}

% Redefine \citep command
\let\oldcitep\citep
\renewcommand{\citep}[1]{%
    \textcolor{citblue}{\oldcitep{#1}}%
}

% Define norm and abs for convenience
\newcommand\norm[1]{\left\lVert#1\right\rVert}
\newcommand\abs[1]{\left\lvert#1\right\rvert}